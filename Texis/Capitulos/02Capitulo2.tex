%---------------------------------------------------------------------
%
%                          Cap�tulo 1
%
%---------------------------------------------------------------------

\chapter{Estado del arte}

\begin{FraseCelebre}
\begin{Frase}
Frase to ingeniosa
\end{Frase}
\begin{Fuente}
Persona ToImportante
\end{Fuente}
\end{FraseCelebre}

\begin{resumen}
En este cap�tulo se van a tratar los aspectos m�s importantes tanto de la computaci�n emocional como de las diferentes tecnolog�as que vamos a utilizar.
\end{resumen}


%-------------------------------------------------------------------
\section{Computaci�n Afectiva}
%-------------------------------------------------------------------
\label{cap2:sec:computacion_afectiva}

La computaci�n emocional es el estudio y el desarrollo de sistemas capaces de percibir e interpretar las emociones humanas.

	\subsection{Diccionarios Afectivos}
	\label{cap2:subsec:diccionarios}
	Diccionarios que existen actualmente.
	
	\subsection{Nuestro diccionario}
	\label{cap2:subsec:nuestro_diccionario}
	El diccionario que vamos a utilizar.

%-------------------------------------------------------------------
\section{Servicios Web}
%-------------------------------------------------------------------
\label{cap2:sec:servicios_web}

Aqu� va la parte de Servicios Web.

%-------------------------------------------------------------------
\section{Metodolog�a Scrum}
%-------------------------------------------------------------------
\label{cap2:sec:scrum}

Explicaci�n metodolog�a Scrum

%-------------------------------------------------------------------
\section{Integraci�n Continua}
%-------------------------------------------------------------------
\label{cap2:sec:integracion}

Explicaci�n Integraci�n Continua

%-------------------------------------------------------------------
\section*{\NotasBibliograficas}
%-------------------------------------------------------------------
\TocNotasBibliograficas

Citamos algo para que aparezca en la bibliograf�a\ldots
\citep{ldesc2e}

\medskip

Y tambi�n ponemos el acr�nimo \ac{CVS} para que no cruja.

Ten en cuenta que si no quieres acr�nimos (o no quieres que te falle la compilaci�n en ``release'' mientras no tengas ninguno) basta con que no definas la constante \verb+\acronimosEnRelease+ (en \texttt{config.tex}).


%-------------------------------------------------------------------
\section*{\ProximoCapitulo}
%-------------------------------------------------------------------
\TocProximoCapitulo

En el pr�ximo cap�tulo se tratar�n los primeros servicios web.

% Variable local para emacs, para  que encuentre el fichero maestro de
% compilaci�n y funcionen mejor algunas teclas r�pidas de AucTeX
%%%
%%% Local Variables:
%%% mode: latex
%%% TeX-master: "../Tesis.tex"
%%% End:
